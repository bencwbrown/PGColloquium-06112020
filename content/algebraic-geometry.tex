\section{Classical Algebraic Geometry}

\begin{frame}{Foray Into Algebraic Geometry}
    \begin{block}{Projective Space}
    Playing field is $n$\emph{-dimensional projective space}, $\PP^{n}$:
        $$ \PP^{n} := \{ (z_{0}, \ldots, z_{n}) \in \CC^{n} \} / (\vb{x} \sim \lambda \cdot \vb{y}), \quad \lambda \neq 0, $$
    that is, its elements consists of \emph{lines through the origin} in $\CC^{n}$.
    \end{block}
    

    \begin{center}
        \includegraphics[height=0.35\textwidth, angle=0]{resources/projective-space}
    \end{center}

\end{frame}

\begin{frame}{Varieties}
    
    \emph{Varieties} are the objects studied in algebraic geometry, determined by the \emph{vanishing set}\footnote{from `\emph{Verschwindungsmenge}'} $V(-)$, for a system of polynomials.

\begin{figure}
        \centering
        \begin{subfigure}[b]{0.475\textwidth}
            \centering
            \includegraphics[width=0.35\textwidth]{resources/two-hyperboloid}
            \caption[]%
            {{\small $V(x^{2} + y^{2} - z^{2} + 1)$}}    
        \end{subfigure}
        \hfill
        \begin{subfigure}[b]{0.475\textwidth}  
            \centering 
            \includegraphics[width=0.35\textwidth]{resources/one-hyperboloid}
            \caption[]%
            {{\small $V(x^{2} + y^{2} - z^{2} - 1)$}}    
        \end{subfigure}
        \hfill
        \begin{subfigure}[b]{0.475\textwidth}   
            \centering 
            \includegraphics[width=0.35\textwidth]{resources/sphere}
            \caption[]%
            {{\small $V(x^{2} + y^{2} + z^{2} - 1)$}}    
        \end{subfigure}
        \hfill
        \begin{subfigure}[b]{0.475\textwidth}   
            \centering 
            \includegraphics[width=0.35\textwidth]{resources/cone}
            \caption[]%
            {{\small $V(x^{2} + y^{2} - z^{2})$}}    
        \end{subfigure}
    \end{figure}

\end{frame}

\begin{frame}{Segre Varieties}
        \emph{Segre varieties} come from $\sigma: \PP^{n} \times \PP^{m} \ra \PP^{(n+1)(m+1) - 1}$, that sends $([X],[Y])$ to the pairwise products of their components:
            $$ \sigma : ([X_{1}, \ldots, X_{n+1}], [Y_{1}, \ldots, Y_{m+1}]) \mapsto [\ldots, X_{i}Y_{j}, \ldots ]. $$

        \begin{block}{Example (Segre quadric surface)}
        \vspace{-24pt}
        $$\sigma : \PP^{1} \times \PP^{1} \ra \PP^{3},\ ([X_{1}, X_{2}], [Y_{1}, Y_{2}]) \mapsto [ X_{1}Y_{1}, X_{1}Y_{2}, X_{2}Y_{1}, X_{2}Y_{2} ]. $$
        
        Set $[ X_{1}Y_{1}, X_{1}Y_{2}, X_{2}Y_{1}, X_{2}Y_{2} ] =: [p_{11}, p_{12}, p_{21}, p_{22}]$, then:
        $$ \rightsquigarrow \det \begin{pmatrix} p_{11} & p_{12} \\ p_{21} & p_{22} \end{pmatrix} = 0 \iff \rank \begin{pmatrix} p_{11} & p_{12} \\ p_{21} & p_{22} \end{pmatrix} \leq 1. $$
        \end{block}
\end{frame}

\begin{frame}
    \begin{block}{Rulings}
    The Segre quadric surface has two families of lines in it, called \emph{rulings}. These are the images of $\sigma(\PP^{1} \times \{ \text{pt} \} )$ and $\sigma( \{ \text{pt} \} \times \PP^{1})$ in $\PP^{3}$.
    \end{block}

    \begin{figure}[!htb]
    \centering
    \begin{minipage}{0.3\textwidth}
        \centering
        \includegraphics[height=0.5\textheight]{resources/shukhov-1.jpg}
        \caption{\emph{Shukhov Tower}, Nizhny Novgorod}
    \end{minipage}%
    \begin{minipage}{0.3\textwidth}
        \centering
        \includegraphics[height=0.5\textheight]{resources/shukhov-2.jpg}
        \caption{\emph{Shukhov Tower}, Moscow}
    \end{minipage}%
    \begin{minipage}{.3\textwidth}
        \centering
        \includegraphics[height=0.5\textheight]{resources/tractricious.jpg}
        \caption{\emph{Tractricious}, Fermilab}
    \end{minipage}
    \end{figure}

\end{frame}

\begin{frame}{Manifold of Independence}

    \begin{itemize}
        \item Let $\Delta_{3} \subset \RR^{4}$, with vertices $A_{i} = e_{i}$, and let $p = (p_{ij}) \in \Delta_{3}$ be
        \vspace{-6pt}
        \begin{center}
        \begin{table}[]
        $p_{ij} = (p_{11}, p_{12}, p_{21}, p_{22}) =$ 
        \begin{tabular}{|l|l|}
        \hline
        $p_{11}$ & $p_{12}$ \\ \hline
        $p_{21}$ & $p_{22}$ \\ \hline
        \end{tabular}
        \end{table}
        \end{center}

    \item Has been shown that the two rulings are given by \cite{SFJG1970}:
    
    \begin{center}
    \begin{table}[]
    $p_{ij}(s,t) =$
    \begin{tabular}{|c|c|}
    \hline
    $st$ & $s(1-t)$ \\ \hline
    $t(1-s)$ & $(1-s)(1-t)$ \\ \hline
    \end{tabular}
    $\quad (0 \leq s,t \leq 1).$
    \end{table}
    \end{center}
    
    \end{itemize}

    \begin{figure}[!htb]
    \centering
    \begin{minipage}{.5\textwidth}
        \centering
        \includegraphics[height=0.35\textheight]{resources/tetrahedron.pdf}
    \end{minipage}%
    \begin{minipage}{0.5\textwidth}
        \centering
        \includegraphics[height=0.40\textheight]{resources/independence-surface.pdf}
    \end{minipage}
    \end{figure}
    
\end{frame}