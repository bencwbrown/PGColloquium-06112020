\section{Introduction}

\begin{frame}{\emph{``Does Watching Football on TV Cause Hair Loss?''}}
    
    In a fictional study, 296 British subjects between the ages of 40 -- 50 were asked about their hair loss and how much football they watch on television.
        
    \begin{block}{$3 \times 3$ Contingency Table}

    We can represent the responses in a $3\times 3$ \emph{contingency table}:

    \begin{table}[h]
        \begin{tabular}{@{}lcccc@{}} 
        & &  \multicolumn{3}{c}{Hair Amount}\\\cmidrule{3-5} 
        TV Time & & lots & medium & balding\\ \midrule 
        $\geq 2$h & & 51 & 45 & 33 \\ 
        $2-6$ h & & 28 & 30 & 29 \\ 
        $\geq 6$h & & 15 & 27 & 38\\\bottomrule
        \end{tabular}
    \end{table} 
    \end{block}

    Based on the data, are these two random variables independent, or is there a correlation?

\end{frame}

\begin{frame}{\emph{``Does Watching Football on TV Cause Hair Loss?''}}
    $$ M = \begin{bmatrix} 
    51 & 45 & 33 \\ 
    28 & 30 & 29 \\ 
    15 & 27 & 38
    \end{bmatrix} $$

    \begin{block}{Null Hypothesis}
    \vspace*{-18pt}
        \begin{equation*}
            H_{0}: \qquad \text{\emph{Football on TV and Hair Loss are Independent.}}
        \end{equation*}
    \vspace*{-18pt}
    \end{block}

    \begin{itemize}
        \item Independence means that the $(2 \times 2)$-minors of the data matrix $M$ should vanish, but in fact they are all strictly quite positive, suggesting a positive correlation!
        \item For example, the top left $(2 \times 2)$-minor equals: $51 \cdot 30 - 45 \cdot 28 = 1530 - 1260 = 270 \neq 0$.
    \end{itemize}

\end{frame}

\begin{frame}{\emph{``Does Watching Football on TV Cause Hair Loss?''}}

    \begin{itemize}
        \item A better explanation for our data is obtained by identifying a certain \emph{hidden variable}, which is the \emph{gender identification} of the respondents:
    \end{itemize}

    $$ M  = M_{m} + M_{f} = 
    \underbrace{\begin{bmatrix} 
    3 & 9 & 15 \\ 
    4 & 12 & 20 \\ 
    7 & 21 & 35
    \end{bmatrix}}_{126 \text{ male}} + 
    \underbrace{\begin{bmatrix} 
    48 & 36 & 18 \\
    24 & 18 & 9 \\
    8 & 6 & 3
    \end{bmatrix}}_{170 \text{ female}}.  $$

    \begin{block}{Alternative Hypothesis}
        Instead, we have \emph{conditional independence}:
        \begin{equation*}
            H_{0}: \text{ \emph{Football on TV \& Hair Loss are Independent} given \emph{Gender.}}
        \end{equation*}
    \end{block}

\end{frame}
