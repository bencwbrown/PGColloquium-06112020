%
% XeTeX specific
%
\RequireXeTeX
\usepackage{xltxtra}
\usepackage{fontspec}
\usepackage{xunicode}

% -----------------------------------
% font config
% -----------------------------------

\setsansfont{Linux Biolinum}
\setromanfont{Linux Libertine}
\setmonofont[Scale=0.9]{Consolas}

% -----------------------------------
% math
% -----------------------------------

\usepackage{amsmath}
\usepackage{amsfonts}
\usepackage{amssymb}
\usepackage{tikz}
\usetikzlibrary{decorations.pathmorphing}
\usepackage{amsthm}
\usepackage{mathrsfs}   % \mathscr
\usepackage{stmaryrd}   % \lightning
\usepackage{physics}
\usepackage{booktabs}
\usepackage{pgfplots}
\pgfplotsset{compat=1.17}
\usepackage{pgfplotstable}

% -----------------------------------
% macros
% -----------------------------------

\newcommand{\ra}{\rightarrow}
\newcommand{\la}{\leftarrow}
\newcommand{\lra}{\longrightarrow}
\newcommand{\lla}{\longleftarrow}
\newcommand{\into}{\hookrightarrow}
\newcommand{\NN}{\mathbb{N}}
\newcommand{\ZZ}{\mathbb{Z}}
\newcommand{\QQ}{\mathbb{Q}}
\newcommand{\RR}{\mathbb{R}}
\newcommand{\CC}{\mathbb{C}}
\newcommand{\PP}{\mathbb{P}}
\newcommand{\half}{\frac{1}{2}}
\newcommand{\Mat}{\text{Mat}}
\newcommand{\Span}{\text{Span}}
\newcommand{\Id}{\operatorname{Id}}
\newcommand{\Td}{\operatorname{Td}}
\newcommand{\Prob}{\operatorname{Prob}}
\newcommand{\Mixt}{\operatorname{Mixt}}
\newcommand{\Sec}{\operatorname{Sec}}
\newcommand{\mc}[1]{\mathcal{#1}}

\newcommand{\indep}{\perp \!\!\! \perp}
\newlength{\LETTERheight}
\newcommand*{\longleadsto}[1]{\ \raisebox{0.24\LETTERheight}{\tikz \draw [->,
line join=round,
decorate, decoration={
    zigzag,
    segment length=4,
    amplitude=.9,
    post=lineto,
    post length=2pt
}] (0,0) -- (#1,0);}\ }

\newcommand{\createcontingencytable}[4]{ %
% #1=table name
% #2=first column name
% #3=new row sum name
% #4=new column sum name
\pgfplotstablecreatecol[
    create col/assign/.code={% In each row ... 
        \def\rowsum{0}
        \pgfmathtruncatemacro\maxcolindex{\pgfplotstablecols-1}
        % ... loop over all columns, summing up the elements
        \pgfplotsforeachungrouped \col in {1,...,\maxcolindex}{
            \pgfmathsetmacro\rowsum{\rowsum+\thisrowno{\col}}
        }
        \pgfkeyslet{/pgfplots/table/create col/next content}\rowsum
    }
]{#3}{#1}%
%
% Transpose the table, so we can repeat the summation step for the columns
\pgfplotstabletranspose[colnames from={#2},input colnames to={#2}]{\intermediatetable}{#1}
%
% Sums for each column
\pgfplotstablecreatecol[
    create col/assign/.code={%
        \def\colsum{0}
        \pgfmathtruncatemacro\maxcolindex{\pgfplotstablecols-1}
        \pgfplotsforeachungrouped \col in {1,...,\maxcolindex}{
            \pgfmathsetmacro\colsum{\colsum+\thisrowno{\col}}
        }
        \pgfkeyslet{/pgfplots/table/create col/next content}\colsum
    }
]{#4}\intermediatetable
%
% Transpose back to the original form
\pgfplotstabletranspose[colnames from=#2, input colnames to=#2]{\contingencytable}{\intermediatetable}
}
%

% -----------------------------------
% grammar and textstyle
% -----------------------------------

\usepackage{polyglossia}
\setdefaultlanguage[variant=british]{english}
\usepackage{csquotes}
% underlining
\usepackage{soul} % ulem redifines \emph, this sucks

% -----------------------------------
% colors
% -----------------------------------

\usepackage{xcolor}
\usepackage{colortbl}

% listing colors
% https://kuler.adobe.com/Color-palette-color-theme-4004722
\definecolor{keyword}{RGB}{239,33,74}
\definecolor{number}{RGB}{243,202,22}
\definecolor{comment}{RGB}{126,158,19}
\definecolor{string}{RGB}{6,129,128}


% -----------------------------------
% links and references
% -----------------------------------

\usepackage{hyperref}
\hypersetup{
    colorlinks=false,
    % hide bookmarks
    pdfpagemode=UseNone,
    % meta
    pdfauthor={\presentationAuthor},
    pdftitle={\presentationTitle}
}
\usepackage[english]{cleveref}

% -----------------------------------
% bibliography and glossaries
% -----------------------------------

\usepackage[
    backend=biber,
    style=alphabetic-verb
    ]{biblatex}
\bibliography{presentation}

\usepackage{glossaries}
\input{glossary}
\makeglossaries

% -----------------------------------
% graphics
% -----------------------------------

\usepackage{graphicx}
\usepackage{tikz}
\usetikzlibrary{shapes.geometric, arrows, shadows, positioning}
\usepackage{adforn} % ornaments, used in titlepage

% -----------------------------------
% beamer theme
% -----------------------------------

% you can't locate the theme in a subfolder without shooting yourself in the knee
\usetheme{alinz}

% -----------------------------------
% listings and pseudocode
% -----------------------------------

\Crefname{lstlisting}{Listing}{Listings}
\crefname{lstlisting}{listing}{listings}

\usepackage{listings}
\lstset{
    basicstyle=\footnotesize\ttfamily\color{lightdark},
    backgroundcolor=\color{blockbg},
    numbers=left,
    %numbersep=6pt,
    numberstyle=\scriptsize\color{granite},
    commentstyle=\sffamily\itshape\color{sea},
    keywordstyle=\bfseries\color{raspberry},
    stringstyle=\itshape\color{lake},
    showstringspaces=false,
    breaklines=true,
    breakatwhitespace=true,
    frame=lr,
    framerule=0pt,
    framesep=6pt,
    captionpos=b
}
% for pseudocode
\usepackage[slide,linesnumbered,algoruled]{algorithm2e}

%\input{misc/javascript}
%\input{misc/html5}
